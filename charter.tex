\documentclass[
11pt, % The default document font size, options: 10pt, 11pt, 12pt
%codirector,
]{charter} 


% El títulos de la memoria, se usa en la carátula y se puede usar el cualquier lugar del documento con el comando \ttitle
\titulo{Dise\~no e implementaci\'on de un prototipo de dron de c\'odigo abierto} 

% Nombre del posgrado, se usa en la carátula y se puede usar el cualquier lugar del documento con el comando \degreename
\posgrado{Carrera de Especialización en Sistemas Embebidos} 
%\posgrado{Carrera de Especialización en Internet de las Cosas} 
%\posgrado{Carrera de Especialización en Inteligencia Artificial}
%\posgrado{Maestría en Sistemas Embebidos} 
%\posgrado{Maestría en Internet de las cosas}
% IMPORTANTE: no omitir titulaciones ni tildación en los nombres, también se recomienda escribir los nombres completos (tal cual los tienen en su documento)
% Tu nombre, se puede usar el cualquier lugar del documento con el comando \authorname
\autor{Ing. Daniel David Albarracin}

% El nombre del director y co-director, se puede usar el cualquier lugar del documento con el comando \supname y \cosupname y \pertesupname y \pertecosupname
\director{Ing. Juan Manuel Cruz}
\pertenenciaDirector{FIUBA} 
\codirector{}
\pertenenciaCoDirector{}


% Nombre del cliente, quien va a aprobar los resultados del proyecto, se puede usar con el comando \clientename y \empclientename
\cliente{Comite evaluador}
\empresaCliente{CESE - FIUBA}
 
\fechaINICIO{20 de agosto de 2024}				%Fecha de inicio de la cursada de GdP \fechaInicioName
\fechaFINALPlan{8 de octubre de 2024} 			%Fecha de final de cursada de GdP
\fechaFINALTrabajo{31 de junio de 2025}			%Fecha de defensa pública del trabajo final


\begin{document}

\maketitle
\thispagestyle{empty}
\pagebreak


\thispagestyle{empty}
{\setlength{\parskip}{0pt}
	\tableofcontents{}
}
\pagebreak


\section*{Registros de cambios}
\label{sec:registro}


\begin{table}[ht]
	\label{tab:registro}
	\centering
	\begin{tabularx}{\linewidth}{@{}|c|X|c|@{}}
		\hline
		\rowcolor[HTML]{C0C0C0}
		Revisión & \multicolumn{1}{c|}{\cellcolor[HTML]{C0C0C0}Detalles de los cambios realizados} & Fecha            \\ \hline
		0        & Creación del documento                                                          & \fechaInicioName \\ \hline
		1        & Se completa hasta el punto 5 inclusive                & {3} de {septiembre} de 2024 \\ \hline
		1.1      & Correcciones hasta el punto 5				 		 & {10} de {septiembre} de 2024 \\ \hline
		2        & Correcciones hasta el punto 5. 
				   Se completa hasta el punto 9 inclusive				 & {15} de {septiembre} de 2024 \\ \hline
		%		  Se puede agregar algo más \newline
		%		  En distintas líneas \newline
		%		  Así                                                    & {día} de {mes} de 202X \\ \hline
		%3      & Se completa hasta el punto 12 inclusive                & {día} de {mes} de 202X \\ \hline
		%4      & Se completa el plan	                                 & {día} de {mes} de 202X \\ \hline

		% Si hay más correcciones pasada la versión 4 también se deben especificar acá
	\end{tabularx}
\end{table}

\pagebreak



\section*{Acta de constitución del proyecto}
\label{sec:acta}

\begin{flushright}
	Buenos Aires, \fechaInicioName
\end{flushright}

\vspace{2cm}

Por medio de la presente se acuerda con el \authorname\hspace{1px} que su Trabajo Final de la \degreename\hspace{1px} se titulará ``\ttitle'' y consistirá en la implementación de un prototipo experimental de un drone de bajo coste. El trabajo tendrá un presupuesto preliminar estimado de 664 horas y un costo estimado de \$ 330.000, con fecha de inicio el \fechaInicioName\hspace{1px} y fecha de presentación pública el \fechaFinalName.

Se adjunta a esta acta la planificación inicial.

\vfill

% Esta parte se construye sola con la información que hayan cargado en el preámbulo del documento y no debe modificarla
\begin{table}[ht]
	\centering
	\begin{tabular}{ccc}
		\begin{tabular}[c]{@{}c@{}}Dr. Ing. Ariel Lutenberg \\ Director posgrado FIUBA\end{tabular} & \hspace{2cm} & \begin{tabular}[c]{@{}c@{}}\clientename \\ \empclientename \end{tabular} \vspace{2.5cm} \\
		\multicolumn{3}{c}{\begin{tabular}[c]{@{}c@{}} \supname \\ Director del Trabajo Final\end{tabular}} \vspace{2.5cm}                                                                                   \\
	\end{tabular}
\end{table}




\section{1. Descripción técnica-conceptual del proyecto a realizar}
\label{sec:descripcion}



	La actualidad ha popularizado el uso y desarrollo de la industria de los drones debido a su versatilidad en amplios ámbitos de aplicación. Su uso puede hallarse en campos de investigación, agricultura, seguridad, adquisición de imágenes, militar e incluso recreativo, por solo enumerar algunos campos.
	
	El presente trabajo, de caracter personal, desarrolla un prototipo de drone experimental de bajo costo con componentes disponibles en el mercado.

	El desarrollo de un modelo prototipo de un drone permite la puesta en marcha, verificación y realización de pruebas de las partes funcionales resguardando la integridad del producto.

	El sistema embebido a desarrollar permitirá realizar la función de controlador de vuelo (FC por sus siglas en inglés). El FC basado en un microcontrolador debe ser el punto central de comunicación entre todos los elementos de entrada y salida. En tal sentido y en base a la placa Núcleo STM32F4 funcionando como FC, se desarrollarán las interfaces de hardware y software para lograr la interacción con los componentes mínimos necesarios para un disponer de drone funcional.

	El mercado de los drones abunda en ofertas de distinta calidad, orientadas principalmente a uso recreativo y a uso profesional. En el primer grupo se encuentran drones de bajo costo y baja calidad, mientras que en el otro segmento las soluciones son cerradas y de precio prohibitivo. 
	
	El presente proyecto intenta solventar una doble necesidad: por un lado disponer de un drone funcional a precio asequible y por otro, dar acceso a la información en forma detallada de su desarrollo para ser modificado, editado y copiado sin fines de lucro. 
	

	En la figura \ref{fig:diagBloques} se presenta el diagrama en bloques del sistema. 
	
	Se observa que el FC es el punto de interconexión entre las entradas (IMU, GPS, UltraSonido, barómetro y Rx) y las salidas (Brushless Engine y Telemetría).

	\begin{figure}[htpb]
		\centering
		\includegraphics[width=.75\textwidth]{./Figuras/Diagrama_Bloques.png}
		\caption{Diagrama en bloques del sistema.}
		\label{fig:diagBloques}
	\end{figure}

	\vspace{25px}



\section{2. Identificación y análisis de los interesados}
\label{sec:interesados}



	\begin{table}[ht]
		%\caption{Identificación de los interesados}
		%\label{tab:interesados}
		\begin{tabularx}{\linewidth}{@{}|l|X|X|l|@{}}
			\hline
			\rowcolor[HTML]{C0C0C0}
			Rol           & Nombre y Apellido & Organización       & Puesto                     \\ \hline
			Cliente       & \clientename      & \empclientename    & Jurado                     \\ \hline
			Responsable   & \authorname       & FIUBA              & Alumno                     \\ \hline
			Orientador    & \supname          & \pertesupname      & Director del Trabajo Final \\ \hline
			Usuario final & Aficionados       & Publico en general &                            \\ \hline
		\end{tabularx}
	\end{table}

	\begin{itemize}
		\item Cliente: a definir.
		\item Director: el Ing. Juan Manual Cruz es un referente en el area de desarrollo de software implementando soluciones con sistemas operativos de tiempo real. Ser\'a quien brindar\'a su experiencia y visi\'on cr\'itica en la modularizacion e interconexi\'on del software.  
		\item Usuario final: todo aficionado a la electr\'onica, el desarrollo de software y en espec\'ifico, el campo de uso de drones que desee hacer uso del proyecto. 
		
	 \end{itemize}
		


\section{3. Propósito del proyecto}
\label{sec:proposito}


	Disponibilizar de forma libre la documentaci\'on, c\'odigo y detalles de implementaci\'on y explicaci\'on te\'orica de un dron funcional basado en el microcontrolador STM32F4 y perif\'ericos asociados. 


\section{4. Alcance del proyecto}
\label{sec:alcance}



	A continuaci\'on se listan los alcances del proyecto en lo concerniente al desarrollo de software, hardware y documentaci\'on. 

	El proyecto incluye:
	\begin{itemize}
		\item Software.
		 	\begin{itemize}
			      \item Desarrollo del firmware para el control del motor sin escobillas trif\'asico.
			      \item Desarrollo del firmware para el m\'odulo del sistemas de posicionamiento global (GPS).
			      \item Desarrollo del firmware para el control de la tarjeta SD.
			      \item Desarrollo del firmware para el control de la unidad de medici\'on inercial (IMU). 
			      \item Desarrollo del firmware para el control del sensor de de ultrasonido.
			      \item Desarrollo del firmware para el control del sensor de presi\'on. 
			      \item Desarrollo del firmware para el control del m\'odulo de control de velocidad.
			      \item Desarrollo del firmware para el control del m\'odulo de distribuci\'on de potencia.
			      \item Desarrollo del firmware para el control del m\'odulo de transmisi\'on de radio frecuencia (Rx RF).
			      \item Desarrollo integral sobre un sistema de tiempo real (FreeRTOS) sobre la controladora de vuelo. 
		      \end{itemize}
		\item Hardware.
		      \begin{itemize}
			      \item Un motor sin escobillas trif\'asico (a definir).
			      \item Un m\'odulo de posicionamiento global modelo GPS-NEO-6m del fabricante U-blox.
			      \item Un m\'odulo interfaz con tarjeta SD de tipo gen\'erico a definir.
			      \item Un sensor para la detecci\'on de ondas de ultrasonido de tipo gen\'erico a definir.
			      \item Un unidad de medici\'on inercial modelo MPU9250 del fabricante a definir
			      \item Un sensor de medici\'on de presi\'on bar\'ometrica modelo BMP280 del fabricante Bosch.
			      \item Un m\'odulo receptor de radio frecuencia modelo FS-IA6R del fabricante FLYSKY (a definir). 
			      \item Un m\'odulo transmisor de radio frecuencia modelo FS-I6x del fabricante FLYSKY (a definir).
				  \item	Desarrollo de un circuito impreso prototipo para el m\'odulo control electr\'onico de de velocidad.
			      \item Desarrollo de un circuito impreso prototipo para el m\'odulo de control de distribuci\'on de potencia.
			      \item Una bateria de tipo Litio Polimero (LiPo)(cantidad de celdas, amperaje y voltaje a definir).
		      \end{itemize}
		\item Documentaci\'on.
		      \begin{itemize}
			      \item Documentaci\'on te\'orica y de implementaci\'on sobre el m\'odulo de posicionamiento global.
			      \item Documentaci\'on te\'orica y de implementaci\'on sobre el control del motor sin escobillas trif\'asico y sobre el modulo de control de velocidad.
			      \item Documentaci\'on te\'orica y de implementaci\'on sobre la unidad de medici\'on inercial.
			      \item Documentaci\'on te\'orica y de implementaci\'on sobre el control de sensor de presi\'on barom\'etrica.
			      \item Documentaci\'on te\'orica y de implementaci\'on sobre el m\'odulo de transimsi\'on de radio frecuencia.
			      \item Esquem\'aticos del desarrollo de PCB para control electr\'onico de velocidad.
			      \item Esquem\'aticos del desarrollo de PCB para control de distribuci\'on de potencia.
			      \item Documentaci\'on de integraci\'on.
		      \end{itemize}

	\end{itemize}
	
	El proyecto no incluye:
	\begin{itemize}
		\item General.
		    \begin{itemize}
			    \item Chasis mec\'anico de soporte.
			    \item Pruebas en vuelo. 
			    \item Transimisi\'on de video.
			    \item No transmitir\'a datos al receptor.
			    \item No se realizar\'an pruebas de compatibilidad electromagn\'etica. 
		    \end{itemize}
		\item Software.
		 	\begin{itemize}
				\item Compatibilidad con diferentes tipos de m\'odulos y fabricantes.
			    \item Compatibilidad con protocolos no desarrollados en el proyecto.
			    \item Portaci\'on a otro sistema de tiempo real.
			    \item Portaci\'on a otro microcontrolador.
		    \end{itemize}
		\item Hardware.
		 	\begin{itemize}
				\item Desarrollo comercial de circuito impreso para el m\'odulo de control electr\'onico de velocidad. 
		    	\item Desarrollo comercial de circuito impreso para el m\'odulo de distribuci\'on de potencia.
		    \end{itemize}		
	\end{itemize}




\section{5. Supuestos del proyecto}
\label{sec:supuestos}


	Para el desarrollo del presente proyecto se realizan los siguientes supuestos:

	\begin{itemize}
		\item Realizaci\'on de an\'alisis de caracter\'isticas y alternativas de componentes.
		\item Los m\'odulos y elementos electr\'onicos ser\'an adquiridos en el mercado local.
		\item Se contar\'a con la orientaci\'on t\'ecnica de un director para el desarrollo de los distintos componentes del software.
		\item Se contar\'a con la orientaci\'on t\'ecnica de un colaborador para el desarrollo de hardware (a definir).
		\item Se dispondr\'a de los medios econ\'omicos para la adquisici\'on de los componentes y m\'odulos electr\'onicos. 
		\item Se dispondr\'a de una carga horaria media de dedicacion exlusiva al desarrollo del proyecto de 16 h semanales. 
	\end{itemize}




\section{6. Requerimientos}
\label{sec:requerimientos}


	Los requerimientos del proyecto se listan a continuaci\'on.

	\begin{enumerate}
		\item Requerimientos funcionales:
		      \begin{enumerate}
			      \item El sistema debe operar integramente demostrando la funcionalidad de un dron prototipo. 
			      \item Tanto el transmisor como el receptor de radio frecuencia deben operar en una banda de frecuencia habilitada para uso ciudadano.
			      \item El usuario final debe poder replicar el proyecto siguiendo la documentacion publicada el repositorio del proyecto.  
		      \end{enumerate}
		\item Requerimientos de software:
			  \begin{enumerate}
			      \item El firmware desarrollado para cada componente de hardware del proyecto brindar\'a una api para su
utilizaci\'on.
				  \item El sistema operativo debe ser de tiempo real.			      
			      \item El sistema operativo debe comandar la interoperabilidad con los diferentes componentes de hardware.
			      \item El sistema operativo debe tener un esquema de prioridades y manejo de interrupciones.
		      \end{enumerate}
		\item Requerimiento de testing:
			 \begin{enumerate}
			      \item El firmware para el control de cada componente de hardware ser\'a testeado acorde a la api desarrollada para su operabilidad.
		     \end{enumerate}
		\item Requerimiento de hardware:
			 \begin{enumerate}
			      \item El microcontrolador debe soportar un sistema operativo en tiempo real.
			      \item El m\'odulo para el control electr\'onico de velocidad debe poder operar en los rangos de potencia exigido por el motor sin escobillas trif\'asico.
			      \item El m\'odulo para la distribuci\'on de potencia debe soportar picos de corriente a necesidad y al mismo tiempo protecci\'on de los componentes conectados.
			      \item El m\'odulo controlador de vuelo desarrollado sobre la placa de desarrollo Nucleo SMT32F4 debe brindar interoperabilidad con los siguientes elementos de hardware:
			      \begin{enumerate}
			      	\item Un motor sin escobillas trif\'asico.
			      	\item Un m\'odulo de posicionamiento global.
			      	\item Un m\'odulo sensor de ultrasonido.
  		      	    \item Un m\'odulo sensor de presi\'on barom\'etrica.
  		      	    \item Una unidad de medici\'on inercial.
  		      	    \item Un m\'odulo transmisor de radio frecuencia.
  		      	    \item Un m\'odulo para resguardo en tarjeta SD.
  		      	    \item Un m\'odulo para control de velocidad del motor.
  		      	    \item Un m\'odulo para el control de potencia del sistema. 
			      \end{enumerate}
		     \end{enumerate}
		\item Requerimientos de documentaci\'on:
		      \begin{enumerate}
				  \item El c\'odigo fuente deber\'a estar publicado en un repositorio de \textit{Github}.			      
			      \item El c\'odigo fuente deber\'a estar documentado.
			      \item La documentaci\'on tanto del c\'odigo fuente como los destalles de implementaci\'on ser\'a subido al repositorio del proyecto.
			      \item Los esquem\'aticos de los circuitos impresos ser\'an subidos al repositorio del proyecto.
		      \end{enumerate}
	\end{enumerate}


\section{7. Historias de usuarios (\textit{Product backlog})}
\label{sec:backlog}

	A continuaci\'on se listan algunas historias de usuarios relacionadas a la necesidades del usuario final:
	
	\begin{enumerate}
		\item ``Como aficionado quiero disponer de la informaci\'on recolectada por el dron en una memoria SD para tener trazabilidad de su funcionamiento."
		
		\textit{Story points}: 13 (complejidad: 3, dificultad: 3, incertidumbre: 3)
		
		\item ``Como aficionado quiero disponer de la documentac\'n necesaria para poder replicar el proyecto."

		\textit{Story points}: 8 (complejidad: 3, dificultad: 3, incertidumbre: 2)
		
		\item ``Como aficionado quiero disponer del c\'odigo fuente utilizado en el desarrollo del proyecto para poder 		modificarlo."

		\textit{Story points}: 5 (complejidad: 1, dificultad: 3, incertidumbre: 1)
		
		\item ``Como aficionado quiero disponer de los esquem\'aticos del dise\~no de los circuitos impresos para poder adaptarlos a mis necesidades."

		\textit{Story points}: 5 (complejidad: 3, dificultad: 1, incertidumbre: 1)
		
		\item ``Como aficionado quiero que el dron tenda detecci\'on de colisiones para evitar que se da\~ne."

		\textit{Story points}: 3 (complejidad: 1, dificultad: 1, incertidumbre: 1)
				
	\end{enumerate}

\section{8. Entregables principales del proyecto}
\label{sec:entregables}

	Los entregables del proyecto son:

	\begin{itemize}
		\item Manual de usuario.
		\item Manual de desarrollo de firmware. 
		\item Diagrama de circuitos esquemáticos.
		\item Código fuente del firmware.
		\item Diagrama de instalación.
		\item Memoria del trabajo final.
	\end{itemize}

\section{9. Desglose del trabajo en tareas}
\label{sec:wbs}

	\begin{enumerate}
		\item Gesti\'on del proyecto. (35 h)
		      \begin{enumerate}
			      \item Planificaci\'on del proyecto. (10 h)
			      \item Documentar la planificaci\'on y gesti\'on del proyecto. (10 h)
			      \item Comunicar los avances del proyecto. (10 h)
			      \item Revisi\'on y control de los avances del proyecto. (5 h)
		      \end{enumerate}
		\item Investigaci\'on sobre drones. (40 h)
			  \begin{enumerate}
			  	  \item Investigaci\'on y selecci\'on de componentes. (10 h)
			  	  \item Investigaci\'on sobre sistemas de posicionamiento global. (10 h)
			  	  \item Investigaci\'on sobre motores trif\'asicos. (10 h)
			  	  \item Investigaci\'on sobre unidad de medici\'on inercial (10 h) 
			  \end{enumerate}
		\item Desarrollo de software. (240 h)
		      \begin{enumerate}
			      \item Desarrollo del firmware para el control del m\'odulo de posicionamiento global. (30 h)
			      \item Desarrollo del firmware para el control del sensor de pres\'on bar\'ometrica. (30 h)
			      \item Desarrollo del firmware para el control de la unidad de medici\'on inercial. (30 h)
			      \item Desarrollo del firmware para el control del sensor de ultrasonido. (30 h)
			      \item Desarrollo del firmware para el control del m\'odulo de recepci\'on de radiofrecuencia. (30 h)
			      \item Desarrollo del firmware para el control del m\'odulo de resguardo en tarjeta SD. (30 h)
			      \item Desarrollo del firmware para el control del m\'odulo de control electr\'onico de velocidad. (30 h)
			      \item Desarrollo del firmware para el control del m\'odulo de distribuci\'on de potencia. (30 h)
		      \end{enumerate}
		\item Desarrollo de hardware (90 h)
		      \begin{enumerate}
				  \item	Desarrollo de un circuito impreso prototipo para el m\'odulo control electr\'onico de de velocidad. (45 h)
			      \item Desarrollo de un circuito impreso prototipo para el m\'odulo de control de distribuci\'on de potencia. (45 h)
		      \end{enumerate}
		\item Pruebas funcionales. (60 h)
		      \begin{enumerate}
			      \item Pruebas de testeo sobre los firmware desarrollados. (20 h)
			      \item Pruebas de testeo sobre el hardware desarrollado. (20 h)
			      \item Pruebas integrales. (20 h)
		      \end{enumerate}
		\item Documentaci\'on. (105 h)
		      \begin{enumerate}
			      \item Redacci\'on del documento sobre la implementaci\'on del m\'odulo de posicionamiento global. (15 h)
			      \item Redacci\'on del documento sobre la implementaci\'on del m\'odulo de detecci\'on por ultrasonido. (15 h)
			      \item Redacci\'on del documento sobre la implementaci\'on del m\'odulo del sensor de presi\'on barom\'etrica. (15 h)
			      \item Redacci\'on del documento sobre la implementaci\'on del m\'odulo de medici\'n inercial. (15 h)
			      \item Redacci\'on del documento sobre la implementaci\'on del m\'odulo de control electr\'onico de velocidad. (15 h)
			      \item Redacci\'on del documento sobre la implementaci\'on del m\'odulo distribuci\'on de potencia. (15 h)
			      \item Redacci\'on del documento sobre la implementaci\'on del m\'odulo de resguardo de telemetr\'ia. (15 h)
		      \end{enumerate}
		\item Cierre del proyecto. (94 h)
		      \begin{enumerate}
			      \item Gesti\'on del proceso de cierre del proyecto. (10 h)
			      \item Elaboraci\'on del informe final. (60 h)
			      \item Elaboraci\'on de la presentaci\'on. (20 h)
			      \item Presentaci\'on del informe final. (4 h)
		      \end{enumerate}
	\end{enumerate}

	Cantidad total de horas: 664.

\section{10. Diagrama de Activity On Node}
\label{sec:AoN}

\begin{consigna}{red}
	Armar el AoN a partir del WBS definido en la etapa anterior.

	Una herramienta simple para desarrollar los diagramas es el Draw.io (\url{https://app.diagrams.net/}).
	\href{https://app.diagrams.net}{Draw.io}


	\begin{figure}[htpb]
		\centering
		\includegraphics[width=.8\textwidth]{./Figuras/AoN.png}
		\caption{Diagrama de \textit{Activity on Node}.}
		\label{fig:AoN}
	\end{figure}

	Indicar claramente en qué unidades están expresados los tiempos.
	De ser necesario indicar los caminos semi críticos y analizar sus tiempos mediante un cuadro.
	Es recomendable usar colores y un cuadro indicativo describiendo qué representa cada color.

\end{consigna}

\section{11. Diagrama de Gantt}
\label{sec:gantt}

\begin{consigna}{red}
	Existen muchos programas y recursos \textit{online} para hacer diagramas de Gantt, entre los cuales destacamos:

	\begin{itemize}
		\item Planner
		\item GanttProject
		\item Trello + \textit{plugins}. En el siguiente link hay un tutorial oficial: \\ \url{https://blog.trello.com/es/diagrama-de-gantt-de-un-proyecto}
		\item Creately, herramienta online colaborativa. \\\url{https://creately.com/diagram/example/ieb3p3ml/LaTeX}
		\item Se puede hacer en latex con el paquete \textit{pgfgantt}\\ \url{http://ctan.dcc.uchile.cl/graphics/pgf/contrib/pgfgantt/pgfgantt.pdf}
	\end{itemize}

	Pegar acá una captura de pantalla del diagrama de Gantt, cuidando que la letra sea suficientemente grande como para ser legible.
	Si el diagrama queda demasiado ancho, se puede pegar primero la ``tabla'' del Gantt y luego pegar la parte del diagrama de barras del diagrama de Gantt.

	Configurar el software para que en la parte de la tabla muestre los códigos del EDT (WBS).\\
	Configurar el software para que al lado de cada barra muestre el nombre de cada tarea.\\
	Revisar que la fecha de finalización coincida con lo indicado en el Acta Constitutiva.

	En la figura \ref{fig:gantt}, se muestra un ejemplo de diagrama de gantt realizado con el paquete de \textit{pgfgantt}.
	En la plantilla pueden ver el código que lo genera y usarlo de base para construir el propio.

	Las fechas pueden ser calculadas utilizando alguna de las herramientas antes citadas. Sin embargo, el siguiente ejemplo
	fue elaborado utilizando
	\href{https://docs.google.com/spreadsheets/d/1fBz8NhSpc4tkkhz3KjJCbh1nR_ltDkfEcZi4tZXduqs}{esta hoja de cálculo}.

	Es importante destacar que el ancho del diagrama estará dado por la longitud del texto utilizado para las tareas
	(Ejemplo: tarea 1, tarea 2, etcétera) y el valor \textit{x unit}. Para mejorar la apariencia del diagrama, es necesario
	ajustar este valor y, quizás, acortar los nombres de las tareas.

	\begin{figure}[htpb]
		\begin{center}
			\begin{ganttchart}[
					time slot unit=day,
					time slot format=isodate,
					x unit=0.038cm,
					y unit title=0.7cm,
					y unit chart=0.6cm,
					milestone/.append style={xscale=4}
				]{2021-03-05}{2021-12-16}
				\gantttitlecalendar*{2021-03-05}{2021-12-16}{year} \\
				\gantttitlecalendar*{2021-03-05}{2021-12-16}{month} \\
				\ganttgroup{Duración Total}{2021-03-05}{2021-12-16} \\
				%%%%%%%%%%%%%%%%%Organización
				\ganttgroup{Organización}{2021-03-05}{2021-04-16} \\
				\ganttbar{Planificación del proyecto}{2021-03-05}{2021-04-15} \\
				%%%%%%%%%%%%%%%%%Ejecución
				\ganttgroup{Ejecución}{2021-04-16}{2021-10-21} \\
				\ganttbar{Tarea 1}{2021-04-16}{2021-04-29} \\
				\ganttbar{Tarea 2}{2021-04-30}{2021-05-13} \\
				\ganttbar{Tarea 3}{2021-05-14}{2021-05-27} \\
				\ganttbar{Tarea 4}{2021-05-28}{2021-07-12} \\
				\ganttbar{Tarea 5}{2021-07-13}{2021-08-09} \\
				\ganttbar{Tarea 6}{2021-08-10}{2021-09-23} \\
				\ganttbar{Tarea 7}{2021-09-24}{2021-09-30} \\
				\ganttbar{Tarea 8}{2021-10-01}{2021-10-14} \\
				\ganttbar{Tarea 9}{2021-10-15}{2021-10-21} \\
				% %%%%%%%%%%%%%%%%%Finalización
				\ganttgroup{Finalización}{2021-10-22}{2021-12-16} \\
				\ganttbar{Memoria v1}{2021-10-22}{2021-11-04} \\
				\ganttbar{Memoria v2}{2021-11-05}{2021-11-18} \\
				\ganttbar{Memoria final}{2021-11-19}{2021-12-02} \\
				% La fecha del siguiente milestone es la fecha en que terminamos la memoria
				\ganttmilestone{Enviar memoria al director}{2021-12-02} \\
				\ganttbar{Elaborar la presentación}{2021-12-03}{2021-12-16} \\
				\ganttmilestone{Ensayo de la presentación}{2021-12-16} \\
				%%%%%%%%%%%%%%%%%%%%%%%%%%%%%%%%%%%%%%%%%%%%%%%%%%%%%%%%%%%%%%%
			\end{ganttchart}
		\end{center}
		\caption{Diagrama de gantt de ejemplo}
		\label{fig:gantt}
	\end{figure}


	\begin{landscape}
		\begin{figure}[htpb]
			\centering
			\includegraphics[height=.85\textheight]{./Figuras/Gantt-2.png}
			\caption{Ejemplo de diagrama de Gantt (apaisado).} %Modificar este título acorde.
			\label{fig:diagGantt}
		\end{figure}

	\end{landscape}

\end{consigna}


\section{12. Presupuesto detallado del proyecto}
\label{sec:presupuesto}

\begin{consigna}{red}
	Si el proyecto es complejo entonces separarlo en partes:
	\begin{itemize}
		\item Un total global, indicando el subtotal acumulado por cada una de las áreas.
		\item El desglose detallado del subtotal de cada una de las áreas.
	\end{itemize}

	IMPORTANTE: No olvidarse de considerar los COSTOS INDIRECTOS.

	Incluir la aclaración de si se emplea como moneda el peso argentino (ARS) o si se usa moneda extranjera (USD, EUR, etc). Si es en moneda extranjera se debe indicar la tasa de conversión respecto a la moneda local en una fecha dada.

\end{consigna}

\begin{table}[htpb]
	\centering
	\begin{tabularx}{\linewidth}{@{}|X|c|r|r|@{}}
		\hline
		\rowcolor[HTML]{C0C0C0}
		\multicolumn{4}{|c|}{\cellcolor[HTML]{C0C0C0}COSTOS DIRECTOS}   \\ \hline
		\rowcolor[HTML]{C0C0C0}
		Descripción                                                 &
		\multicolumn{1}{c|}{\cellcolor[HTML]{C0C0C0}Cantidad}       &
		\multicolumn{1}{c|}{\cellcolor[HTML]{C0C0C0}Valor unitario} &
		\multicolumn{1}{c|}{\cellcolor[HTML]{C0C0C0}Valor total}        \\ \hline
		                                                            &
		\multicolumn{1}{c|}{}                                       &
		\multicolumn{1}{c|}{}                                       &
		\multicolumn{1}{c|}{}                                           \\ \hline
		                                                            &
		\multicolumn{1}{c|}{}                                       &
		\multicolumn{1}{c|}{}                                       &
		\multicolumn{1}{c|}{}                                           \\ \hline
		\multicolumn{1}{|l|}{}                                      &
		                                                            &
		                                                            &
		\\ \hline
		\multicolumn{1}{|l|}{}                                      &
		                                                            &
		                                                            &
		\\ \hline
		\multicolumn{3}{|c|}{SUBTOTAL}                              &
		\multicolumn{1}{c|}{}                                           \\ \hline
		\rowcolor[HTML]{C0C0C0}
		\multicolumn{4}{|c|}{\cellcolor[HTML]{C0C0C0}COSTOS INDIRECTOS} \\ \hline
		\rowcolor[HTML]{C0C0C0}
		Descripción                                                 &
		\multicolumn{1}{c|}{\cellcolor[HTML]{C0C0C0}Cantidad}       &
		\multicolumn{1}{c|}{\cellcolor[HTML]{C0C0C0}Valor unitario} &
		\multicolumn{1}{c|}{\cellcolor[HTML]{C0C0C0}Valor total}        \\ \hline
		\multicolumn{1}{|l|}{}                                      &
		                                                            &
		                                                            &
		\\ \hline
		\multicolumn{1}{|l|}{}                                      &
		                                                            &
		                                                            &
		\\ \hline
		\multicolumn{1}{|l|}{}                                      &
		                                                            &
		                                                            &
		\\ \hline
		\multicolumn{3}{|c|}{SUBTOTAL}                              &
		\multicolumn{1}{c|}{}                                           \\ \hline
		\rowcolor[HTML]{C0C0C0}
		\multicolumn{3}{|c|}{TOTAL}                                 &
		\\ \hline
	\end{tabularx}%
\end{table}


\section{13. Gestión de riesgos}
\label{sec:riesgos}

\begin{consigna}{red}
	a) Identificación de los riesgos (al menos cinco) y estimación de sus consecuencias:

	Riesgo 1: detallar el riesgo (riesgo es algo que si ocurre altera los planes previstos de forma negativa)
	\begin{itemize}
		\item Severidad (S): mientras más severo, más alto es el número (usar números del 1 al 10).\\
		      Justificar el motivo por el cual se asigna determinado número de severidad (S).
		\item Probabilidad de ocurrencia (O): mientras más probable, más alto es el número (usar del 1 al 10).\\
		      Justificar el motivo por el cual se asigna determinado número de (O).
	\end{itemize}

	Riesgo 2:
	\begin{itemize}
		\item Severidad (S): X.\\
		      Justificación...
		\item Ocurrencia (O): Y.\\
		      Justificación...
	\end{itemize}

	Riesgo 3:
	\begin{itemize}
		\item Severidad (S):  X.\\
		      Justificación...
		\item Ocurrencia (O): Y.\\
		      Justificación...
	\end{itemize}


	b) Tabla de gestión de riesgos:      (El RPN se calcula como RPN=SxO)

	\begin{table}[htpb]
		\centering
		\begin{tabularx}{\linewidth}{@{}|X|c|c|c|c|c|c|@{}}
			\hline
			\rowcolor[HTML]{C0C0C0}
			Riesgo & S & O & RPN & S* & O* & RPN* \\ \hline
			       &   &   &     &    &    &      \\ \hline
			       &   &   &     &    &    &      \\ \hline
			       &   &   &     &    &    &      \\ \hline
			       &   &   &     &    &    &      \\ \hline
			       &   &   &     &    &    &      \\ \hline
		\end{tabularx}%
	\end{table}

	Criterio adoptado:

	Se tomarán medidas de mitigación en los riesgos cuyos números de RPN sean mayores a...

	Nota: los valores marcados con (*) en la tabla corresponden luego de haber aplicado la mitigación.

	c) Plan de mitigación de los riesgos que originalmente excedían el RPN máximo establecido:

	Riesgo 1: plan de mitigación (si por el RPN fuera necesario elaborar un plan de mitigación).
	Nueva asignación de S y O, con su respectiva justificación:
	\begin{itemize}
		\item Severidad (S*): mientras más severo, más alto es el número (usar números del 1 al 10).
		      Justificar el motivo por el cual se asigna determinado número de severidad (S).
		\item Probabilidad de ocurrencia (O*): mientras más probable, más alto es el número (usar del 1 al 10).
		      Justificar el motivo por el cual se asigna determinado número de (O).
	\end{itemize}

	Riesgo 2: plan de mitigación (si por el RPN fuera necesario elaborar un plan de mitigación).

	Riesgo 3: plan de mitigación (si por el RPN fuera necesario elaborar un plan de mitigación).

\end{consigna}


\section{14. Gestión de la calidad}
\label{sec:calidad}

\begin{consigna}{red}
	Elija al menos diez requerimientos que a su criterio sean los más importantes/críticos/que aportan más valor y para cada uno de ellos indique las acciones de verificación y validación que permitan asegurar su cumplimiento.

	\begin{itemize}
		\item Req \#1: copiar acá el requerimiento con su correspondiente número.

		      \begin{itemize}
			      \item Verificación para confirmar si se cumplió con lo requerido antes de mostrar el sistema al cliente. Detallar.
			      \item Validación con el cliente para confirmar que está de acuerdo en que se cumplió con lo requerido. Detallar.
		      \end{itemize}

	\end{itemize}

	Tener en cuenta que en este contexto se pueden mencionar simulaciones, cálculos, revisión de hojas de datos, consulta con expertos, mediciones, etc.

	Las acciones de verificación suelen considerar al entregable como ``caja blanca'', es decir se conoce en profundidad su funcionamiento interno.

	En cambio, las acciones de validación suelen considerar al entregable como ``caja negra'', es decir, que no se conocen los detalles de su funcionamiento interno.

\end{consigna}

\section{15. Procesos de cierre}
\label{sec:cierre}

\begin{consigna}{red}
	Establecer las pautas de trabajo para realizar una reunión final de evaluación del proyecto, tal que contemple las siguientes actividades:

	\begin{itemize}
		\item Pautas de trabajo que se seguirán para analizar si se respetó el Plan de Proyecto original:\\
		      - Indicar quién se ocupará de hacer esto y cuál será el procedimiento a aplicar.
		\item Identificación de las técnicas y procedimientos útiles e inútiles que se emplearon, los problemas que surgieron y cómo se solucionaron:\\
		      - Indicar quién se ocupará de hacer esto y cuál será el procedimiento para dejar registro.
		\item Indicar quién organizará el acto de agradecimiento a todos los interesados, y en especial al equipo de trabajo y colaboradores:\\
		      - Indicar esto y quién financiará los gastos correspondientes.
	\end{itemize}

\end{consigna}

\end{document}